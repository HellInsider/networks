Рассмотрим пример работы алгоритма на модельном случае с $n = 5, \; m = 1$. В качестве индексов сопоставим генералам числа от 0 до 4 включительно. Последний генерал будет византийским, остальные -- честными. Честным генералам изначально сопоставим значения вида $t_i$, где $i$ -- индекс генерала. Византийский генерал будет на первом этапе отправлять значения вида $m3_i$, где $i$ -- индекс генерала, которому адресовано сообщение, а на втором шаге -- $m3_{ij}$, где $i$ -- индекс генерала, которому адресовано сообщение, $j$ -- индекс генерала, от которого (как утверждает византийский генерал) было получено это значение на первом этапе.

По результатам первого этапа генералами были сформированы следующие вектора:

\begin{figure}[H]
	\begin{center}
		\includegraphics[scale=1.0]{5_1}
		\caption{$n = 5, m = 1$. Первый этап}
	\end{center}
\end{figure}

На втором шаге получены следующие параметры:

\begin{figure}[H]
	\begin{center}
		\includegraphics[scale=0.8]{5_2}
		\caption{$n = 5, m = 1$. Второй этап}
	\end{center}
\end{figure}

Затем путём выбора наиболее часто встречающегося элемента, генералы сформировали следующие результаты:

\begin{figure}[H]
	\begin{center}
		\includegraphics[scale=1.0]{5_3}
		\caption{$n = 5, m = 1$. Третий этап}
	\end{center}
\end{figure}

Можно заметить, что результаты у всех честных генералов совпадают, а также значения, полученные для честных генералов, соответствуют их реальным значениям (для византийского генерала значение в итоге оказалось неопределённым, так как на первом этапе он всем генералам рассылал разные значения). Можем сделать вывод, что задача византийских генералов решена корректно.

Тем не менее, у алгоритма есть ограничения. Например, если рассмотреть аналогичный случай при $n = 3$ и $m = 1$, византийским опять будет последний генерал, с индексом 2. 

\begin{figure}[H]
	\begin{center}
		\includegraphics[scale=1.0]{3}
		\caption{$n = 3, m = 1$}
	\end{center}
\end{figure}

Честным генералам удалось достичь формального консенсуса, так как их результирующие вектора совпадают (только при условии, что они ``забывают'' своё собственное значение, и пытаются восстановить его, действуя по протоколу), но при этом получить достоверную информацию о значениях друг друга честным генералам не удалось.
