Генералы будут общаться по протоколу, соответствующему частному случаю алгоритма Лампорта-Шостака-Пиза. Обмен сообщениями будет происходить в 2 этапа:
\begin{itemize}
	\item На первом этапе каждый генерал передаёт всем остальным одно значение, при этом невизантийские генералы честно передают своё значение $v_i$, а византийские могут передавать произвольное значение (при этом он может передавать разным генералам разные значения). В результате у каждого генерала образуется вектор значений, пришедших ему от остальных
	\item На втором этапе каждый невизантийский генерал передаёт всем остальным вектор значений, сформированный на первом этапе, а византийский – вектор произвольных значений (потенциально различных для различных генералов)
\end{itemize}
    
В результате у каждого генерала формируется матрица информации, состоящая из вектора, сформированного на первом этапе, и векторов, полученных на втором этапе.
Таким образом у генерала про каждого союзника формируется набор из нескольких (потенциально различных) значений. В качестве итогового значения, генерал выбирает наиболее часто встречающееся в наборе. Если таких значений несколько, то итоговое значение считается неопределенным.
Алгоритм Лампорта-Шостака-Пиза гарантирует, что следуя его протоколу генералы всегда смогу прийти к консенсусу, в случае если $n > 3m$.