\documentclass[14pt,a4paper,article]{ncc}
\usepackage[a4paper, mag=1000, left=2.5cm, right=1cm, top=2cm, bottom=2cm, headsep=0.7cm, footskip=1cm]{geometry}
\usepackage[utf8]{inputenc}
\usepackage[T2A]{fontenc}
\usepackage[english,russian]{babel}
\usepackage{indentfirst}
%\usepackage[dvipsnames]{xcolor}
\usepackage{amsfonts} 
\usepackage{amssymb} 
\usepackage{amsmath, etoolbox}
\usepackage{graphicx}
\usepackage{float}
\graphicspath{{../figure/}}
\DeclareGraphicsExtensions{.png,.jpg, .jpeg}

%\bibliographystyle{gost-numeric.bbx}
\usepackage{csquotes}
\usepackage[backend=biber]{biblatex}
\addbibresource{literature.bib}

\usepackage{fancyhdr}
\pagestyle{fancy}
\fancyhead[LE,RO]{\thepage}
\fancyfoot{} 

\usepackage{listings}

%\patchcmd\subequations
%{\theparentequation\alph{equation}}
%{\subequationsformat}
%{}{}

%\newcommand{\subequationsformat}{\theparentequation.\arabic{equation}}

%\numberwithin{equation}{subsection}


\usepackage[colorlinks]{hyperref}
\hypersetup{linkcolor=black}

\begin{document}

% Title page 
\begin{titlepage}
    \begin{center}
        \textsc{
            Санкт-Петербургский политехнический университет Петра Великого \\[5mm]
            Физико-механический институт\\[2mm]
            Высшая школа прикладной математики и вычислительной физики
        }   
        \vfill
        \textbf{\large
            Компьютерные сети\\
            Отчёт по лабораторной работе №2 \\
            ``Реализация протокола маршрутизации Open Shortest Path First'' \\[3mm]
            %по курсовой работе \\[3mm]
        }                
    \end{center}

    \vfill
    \hfill
    \begin{minipage}{0.5\textwidth}
        Выполнил: \\[2mm]   
		Студент: Игнатьев Даниил \\
		Группа: 5040102/20201\\
    \end{minipage}

	\hfill
	\begin{minipage}{0.5\textwidth}
		Принял: \\[2mm]
		к. ф.-м. н., доцент \\   
		Баженов Александр Николаевич
	\end{minipage}

    \vfill
    \begin{center}
        \theyear\ г.
    \end{center}
\end{titlepage}

\tableofcontents
%\listoffigures
%\listoftables
\newpage

\section{Постановка задачи}
Требуется реализовать протокол маршрутизации OSPF и проверить работоспособность протокола для следующих видов топологии: линейная, кольцевая, звёздная. Проверить возможность перестройки таблиц достижимости в случае стохастического разрыва связи.

\section{Теория}
Генералы будут общаться по протоколу, соответствующему частному случаю алгоритма Лампорта-Шостака-Пиза. Обмен сообщениями будет происходить в 2 этапа:
\begin{itemize}
	\item На первом этапе каждый генерал передаёт всем остальным одно значение, при этом невизантийские генералы честно передают своё значение $v_i$, а византийские могут передавать произвольное значение (при этом он может передавать разным генералам разные значения). В результате у каждого генерала образуется вектор значений, пришедших ему от остальных
	\item На втором этапе каждый невизантийский генерал передаёт всем остальным вектор значений, сформированный на первом этапе, а византийский – вектор произвольных значений (потенциально различных для различных генералов)
\end{itemize}
    
В результате у каждого генерала формируется матрица информации, состоящая из вектора, сформированного на первом этапе, и векторов, полученных на втором этапе.
Таким образом у генерала про каждого союзника формируется набор из нескольких (потенциально различных) значений. В качестве итогового значения, генерал выбирает наиболее часто встречающееся в наборе. Если таких значений несколько, то итоговое значение считается неопределенным.
Алгоритм Лампорта-Шостака-Пиза гарантирует, что следуя его протоколу генералы всегда смогу прийти к консенсусу, в случае если $n > 3m$.

\section{Реализация}
Протоколы и эмуляторы исполнителей реализованы в двух отдельных потоках на языке Python. Обмен данными осуществляется через очередь сообщений. Программа состоит из следующих элементов:

\begin{itemize}
	\item Sender -- отправитель, формирует сообщения с данными
	\item Reciever -- получатель, получает сообщения и сообщает о факте
	доставки
	\item MsgQueue -- канал коммуникации, который хранит сообщения между
	отправкой и получением, а также имитирует их потерю
\end{itemize}

Каждый пакет содержит информацию о своем порядковом номере в
окне, уникальный номер блока, а также свой статус (доставлен, потерян).
Система принимает следующие параметры:

\begin{itemize}
	\item protocol -- протокол связи
	\item window\_size -- величина окна в выбранном протоколе
	\item timeout -- время в секундах, после которого пакет считается утерянным в случае отсутствия подтверждения его доставки
	\item loss\_probability -- вероятность потери сообщения при передаче [0, 1]
\end{itemize}

\section{Результаты}
Рассмотрим пример работы программы для линейной топологии с 3 узлами. Здеь и далее: узлы – указаны их номера, связи – список номеров соседних узлов на позиции текущего узла.

\begin{itemize}
	\item Узлы [0, 1, 2]
	\item Связи [[1], [0, 2], [1]]
\end{itemize}

К сети подключены все 3 узла. Кратчайшие пути:
\begin{itemize}
	\item 0: [[0], [0, 1], [0, 1, 2]]
	\item 1: [[1, 0], [1], [1, 2]]
	\item 2: [[2, 1, 0], [2, 1], [2]]
\end{itemize}

От сети отключен 2-ой узел. Новые кратчайшие пути:
\begin{itemize}
	\item 0: [[0], [0, 1], []]
	\item 1: [[1, 0], [1], []]
	\item 2: [[], [], [2]]
\end{itemize}

Теперь рассморим пример работы программы для кольцевой топологии с 3 узлами.
\begin{itemize}
	\item Узлы [0, 1, 2]
	\item Связи [[2, 1], [0, 2], [1, 0]]
\end{itemize}

К сети подключены все 3 узла. Кратчайшие пути:
\begin{itemize}
	\item 0: [[0], [0, 1], [0, 2]]
	\item 1: [[1, 0], [1], [1, 2]]
	\item 2: [[2, 0], [2, 1], [2]]
\end{itemize}

От сети отключен 1-ый узел. Новые кратчайшие пути:
\begin{itemize}
	\item 0: [[0], [], [0, 2]]
	\item 1: [[], [1], []]
	\item 2: [[2, 0], [], [2]]
\end{itemize}


Наконец, рассмотрим пример работы программы для звездной топологии с 4 узлами. Центр в узле с индексом 1.

\begin{itemize}
	\item Узлы [0, 1, 2, 3]
	\item Связи [[1], [0, 2, 3], [1], [1]]
\end{itemize}

От сети отключен 3-ий узел. Новые кратчайшие пути:
\begin{itemize}
	\item 0: [[0], [0, 1], [0, 1, 2], []]
	\item 1: [[1, 0], [1], [1, 2], []]
	\item 2: [[2, 1, 0], [2, 1], [2], []]
	\item 3: [[], [], [], [3]]
\end{itemize}


\section{Обсуждение}
В результате работы реализован алгоритм Лампорта-Шостака-Пиза для решения частного случая задачи Византийских генералов. Показана работоспособность алгоритма для $n = 5$ честных генералов и $m = 1$  византийского генерала среди них. Реализована модель взаимодействия между генералами (независимыми узлами) на сетевом и канальном уровне. Для обеспечения корректной работы параллельного алгоритма были использованы различные примитивы синхронизации.

\printbibliography
%\addcontentsline{toc}{section}{Литература}

\section{Приложения} \label{app}

\begin{enumerate}
	\item Репозиторий с кодом программы и кодом отчёта:
	
	\href{https://github.com/HellInsider/networks}{Github}

\end{enumerate}


\end{document}
